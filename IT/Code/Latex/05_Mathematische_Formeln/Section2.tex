\section{Aufgaben}

\subsection{Arbeitsblatt Aufgabe 1}
\begin{enumerate}
\item $f(x)=2x^3-4x^2+1$
\item Die Formel $(3x+4)$ und $(5x^2-7)$
\item $f(x)=\dfrac{5x^4+2x^3-x+1}{4x^3-7}$
\item Die Eulersche Identität $e^{i \cdot \pi}+1=0$
\item Das Integral $\int_a^b \limits f(x)dx = [F(x)]_a^b = |F(b)-F(a)|$
\end{enumerate}c

\subsection{Schnelle Aufgaben}
\begin{itemize}
\item $U_1 = R_1 \cdot I_1$
\item $\alpha_1 = \beta_2 + \gamma_4$
\end{itemize}

\subsection{Array}

$\begin{array}{ll}
\textbf{Behauptung:} 	& $11=7$ \\
\textbf{Beweis:} 		& $Sei $x = 11$ und $y = 7$, dann:$ \\
\end{array}$

$
\begin{array}{lll}
			& x+y=18 								& | \cdot (x-y) 	\\
\Rightarrow & (x+y) \cdot (x-y) = 18 \cdot (x-y) 	& 					\\
\Rightarrow & x^2-y^2=18x-18y						& |-18x\qquad| +y^2 \\
\Rightarrow & x^2-18x=y^2-18y						& |+9^2 			\\
\Rightarrow & x^2-18x+9^2=y^2-18y+9^2				& 					\\
\Rightarrow & (x-9)^2=(y-9)^2						& |\sqrt{\qquad} 	\\
\Rightarrow & x-9=y-9								& | -9 				\\
\Rightarrow & y=9 \mbox{, also } 11=7				& \mathbf{q.e.d.}	\\
\end{array}
$
\end{itemize}