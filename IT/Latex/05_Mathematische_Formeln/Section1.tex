\section{Beispiele}

\subsection{Kraft}
Die Kraft auf einen Körper wird mit $F_{Zug}=m \cdot a$ dargestellt.

\subsection{Binomische Formel}
\begin{equation}
a^2+2ab+b^2
\end{equation}

\subsection{Formel mit Bruch}
\begin{equation}
f(x) = \frac{5x^4-3x^2+7}{\sqrt{3x^2-7x+5}} \cdot e^{jwt}
\end{equation}

\subsection{Grenzwerte}
\begin{equation}
\lim\limits_{n \rightarrow \infty}{a_n}
\end{equation}

\subsection{Produkte, Summen, Integral}
\begin{equation}
\int_0^{2\pi} \limits \sin t\,dt=0
\end{equation}

\subsection{Wurzel}
\begin{equation}
\sqrt{123456}
\end{equation}
\begin{equation}
\sqrt[6]{987654}
\end{equation}
\begin{equation}
\sqrt{\qquad}
\end{equation}
